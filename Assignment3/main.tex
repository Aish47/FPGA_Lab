\documentclass[10pt,a4paper]{article}
\usepackage[utf8]{inputenc}
\usepackage{amsmath}
\usepackage{amsfonts}
\usepackage{multicol}
\usepackage{amssymb}
\usepackage[framed]{matlab-prettifier}
\usepackage{graphicx}
\usepackage[margin=0.75in]{geometry}



\begin{document}
\begin{center}

{\huge EE5811 : FPGA LAB}\\
{\large ASSIGNMENT 3}

\end{center}
P V Aishwarya \\ IS21MTECH14004

\vspace{15pt}
\hrule
\vspace{5pt}


\section*{Problem}

Question 5) c) from papers/icse/cs/2018.pdf\\
Simplify the following expression using Boolean Laws :

\begin{center}
    A.( A' + B ).C.( A + B )
\end{center}

Implement above program in Arduino using assembly language.

\vspace{15pt}
\hrule
\vspace{5pt}

\section*{Solution}

\section*{Truth Table}
\begin{table}[h]
    \centering
    \begin{tabular}{|c|c|c|c|c|}
        \hline
        $A$&$B$&$C$&LHS&RHS \\
        \hline
         0&0&0&0&0 \\
         0&0&1&0&0 \\
         0&1&0&0&0 \\
         0&1&1&0&0 \\
         1&0&0&0&0 \\
         1&0&1&0&0 \\
         1&1&0&0&0 \\
         1&1&1&1&1 \\
         \hline
    \end{tabular}
    \caption{Truth table for A.( A' + B ).C.( A + B ) = A.B.C}
    \label{tab:my_label}
\end{table}

Implemented the above truth table in Arduino. The inputs A,B,C equivalent are displayed on seven segment display and its corresponding output is displayed with LED.\\
Steps: \\
1. Login into ubuntu and go to avra-1.3.0 folder\\
2. In avra-1.3.0 folder open src folder and write program in assign3.asm\\
3. One program was written to display LHS truth table and other to display RHS truth table.\\
4. Compile the program to generate the hex file.\\
5. After generating the hex file save it on laptop and load it in Arduino usng XLoader.\\
\end{document}